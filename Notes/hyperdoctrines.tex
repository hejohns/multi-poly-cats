\documentclass{article}
\usepackage{mathpartir}
\usepackage{tikz-cd}
\usepackage{enumitem}
\usepackage{wrapfig}
\usepackage{fancyvrb}
\usepackage{comment}

\usepackage{ stmaryrd }
\usepackage{ amsmath }


\newcommand{\pipe}{\,|\,}
\newcommand{\rep}{\mathop{\textrm{rep}}}
\newcommand{\downsets}{\mathcal P^{\downarrow}}
\newcommand{\upsets}{\mathcal P^{\uparrow}}
\newcommand{\Poset}{\textrm{Poset}}
\newcommand{\Set}{\textrm{Set}}
\newcommand{\Prop}{\textrm{Prop}}

\begin{document}

\title{First-order Hyperdoctrines}

We give an alternate definition of first-order hyperdoctrines that
makes the Beck-Chevalley condition follow naturally. The key is to
formulate the definitions using the internal language of the category
of presheaves. This is based on the fact that, like most algebraic
structures a functor from $C$ to the category of Heyting algebras is
equivalent to a Heyting algebra internal to the category of functors
$C \to \Set$


\section{Order-theory preliminaries}

Let $f : P \to Q$ be a monotone function of posets. A \textbf{right
  adjoint} to $f$ is a function $g : Q_0 \to P_0$ such that
\[ p \leq g(q) \iff f(p) \leq q \]
A \textbf{left adjoint} is defined dually as satisfying
\[ g(q) \leq p \iff q \leq f(p) \]
Note that in either case $g$ is unique if it exists. The analogous
notion for preorders is unique up to order equivalence.

For any poset $P$ and type $X$ we can define the poset $P^X$ of
functions $X \to P_0$ with the pointwise ordering. We can define a
monotone function $\Delta_X : P \to P^X$
\[ \Delta_X(p)(x) = p \]
which we could sensibly call ``weakening''.
We say $P$ has \textbf{$X$-indexed meets} if $\Delta_X$ has a right
adjoint and \textbf{$X$-indexed joins} if $\Delta_X$ has a left
adjoint. Examples: binary meets and joins are indexed by a two-element
set and top and bottom are indexed by the empty set.

Not all properties are naturally formulated as adjoints. The following
is a generalization.  Let $P$ be a poset. A \emph{downward-closed
subset} or \emph{downset} is a subset $X \subseteq P$ such that if $p
\leq q$ and $q \in X$ then $p \in X$. An upper closed set/upset is
dual. A \emph{representing element} of a downset is a greatest element
of the downset and a representing element of an upset is a least
element of the upset. In this case we say the downset/upset is
\emph{representable}, the idea is that if $x \in X$ is the greatest
element of the downset $X$ .
\[ p \in X \iff p \leq x \]
so $x$ ``represents'' by its inequality predicate the subset.

Let $P$ be a poset with a greatest element $\top$ and $X$ be a
set. Then an \emph{equality function for $X$ in $P$} is a function $=$
in $(P^X)^X$ satisfying
\[ (\forall x,y. (x = y) \leq f(x,y)) \iff \forall x. \top \leq f(x,x) \]
the Leibniz/Lawvere formulation of equality. This is a
representability condition for the upset defined as
\[ f(x,y) \in \textrm{U} \iff \forall x. \top \leq f(x,x) \]

\section{Internal Order Theory}

Let $C$ be a category, then the category of presheaves $\Set^{C^{op}}$
inherits almost all nice properties of the category of sets. One way
to say this is that it is always a \emph{topos}, a model of
intuitionistic type theory. So order theory internal to presheaves
means ``just'' interpret everything in the previous section in this
model of intuitionistic type theory.

But here I'll unravel some of the definitions with an eye to how they
could be formalized in Agda. The definitions in this chapter should
all be interpreted with the modifier ``internal to presheaves on
$C$''.

A \textbf{poset} $X$ consists of
\begin{enumerate}
\item A presheaf $X_0 \in \Set^{C^{op}}$. Below I'll write the
  presheaf operation $X_0 \gamma x$ as $x \circ \gamma$ or simply
  $x\gamma$. We typically suppress the $0$ subscript.
\item A relation $\leq^X : \forall \Gamma \to X_0 \Gamma \to X_0\Gamma
  \to \Prop$ which we write infix as $x \leq^X_\Gamma x'$ (and
  typicall suppress the $X$ superscript) such that for any $\gamma :
  \Delta \to \Gamma$, if $x \leq_\Gamma^X y$ then $x\gamma
  \leq_\Delta^X y\gamma$.
\item Reflexivity and transitivity at each fixed $\Gamma$.
\item Antireflexivity/univalence at each fixed $\Gamma$.
\end{enumerate}
Note that this is definitionally isomorphic to a functor $C^{op}$ to
$\Poset$.

A monotone function from $X$ to $Y$ is
\begin{enumerate}
\item A family of functions $f_\Gamma : X_\Gamma \to Y_\Gamma$ where
  we suppress the subscript if it's clear from context.
\item that is monotone for each $\Gamma$: if $x \leq_\Gamma x'$ then $f(x) \leq_\Gamma f(x')$
\item that commutes with reindexing: $f(x)\gamma = f(x\gamma)$
\end{enumerate}
Viewing the posets as functors this is definitionally isomorphic to a
natural transformation.

For each poset we can define the opposite poset, the terminal poset
and the product poset all by performing the operation on posets
point-wise.

A more interesting operation is \emph{powering}\footnote{Note that
technically terminal and product posets could be defined this way by
using the discrete presheaves $0_\Gamma = \emptyset$ and $2_\Gamma =
\{ 0,1\}$ but the definition simplifies considerably if we define them
directly} a presheaf-internal poset $X$ by a presheaf $A$. We call
this internal poset $X^A$ and it is defined as
\begin{enumerate}
\item The elements are given by the internal hom in the presheaf
  category. This can be defined explicitly as

  \[ X^A(\Gamma) = \Set^{C^{op}}(Y\Gamma \times A, X) \]
\item We give this a point-wise ordering:
  \[ f \leq_{\Gamma} g \iff \forall \Delta, \gamma : C(\Delta,\Gamma), a \in A_\Delta.
  f_\Delta(\gamma,a) \leq_{\Delta} g_\Delta(\gamma,a)
  \]
  Which satisfies the reindexing condition because if $f\leq_\Gamma g$ then
  \begin{align*}
    f \gamma \leq_{\Delta} g\gamma
    &\iff \forall \Theta, \delta : C(\Theta,\Delta), a \in A_\Theta.
    (f\gamma)(\delta, a) \leq_\Theta (g\gamma)(\delta,a)\\
    &\iff \forall \Theta, \delta : C(\Theta,\Delta), a \in A_\Theta.
    f(\gamma\delta, a) \leq_\Theta (g)(\gamma\delta,a)\\
    &\Leftarrow \forall \Theta, \gamma' : C(\Theta,\Gamma), a \in A_\Theta.
    f(\gamma', a) \leq_\Theta (g)(\gamma',a)\\
    &\iff f\leq_\Gamma g
  \end{align*}
\end{enumerate}

A downset\footnote{This could also be defined by defining a poset of propositions and an internal hom of posets} $S$ on a poset $X$ consists of
\begin{enumerate}
\item $S : \forall \Gamma \to X \Gamma \to \Prop$. We write this
  suggestively as $x \in_{\Gamma} S$ or just $x \in S$.
\item that is down closed for each $\Gamma$: if $x \in_\Gamma S$ and
  $y \leq_\Gamma x$ then $y \in_\Gamma S$.
\item and closed under reindexing $\Gamma$, i.e. if $\gamma : \Delta
  \to \Gamma$ and $x \in_\Gamma S$ then $x\gamma \in_\Delta S$
\end{enumerate}
This definition is simple to adapt to Agda. On paper a more abstract
definition in terms of functors would be that it's a functor from
$C^{op}$ to a category of Downsets such that each downset is over the
appropriate poset. Unfortunately I don't see how to do that in such a
way that the downset is definitionally ``over'' the right poset.

Downsets over $X$ define a poset $\downsets X$ with ordering given by
implication. We can define the Yoneda embedding $Y : X \to \downsets X$ by
\[ y \in_\Gamma Y x \iff y \leq_\Gamma x \]

We say that a monotone function $f : X \to \downsets Y$ is
\textbf{representable} if there exists $f' : X \to Y$ such that $f = Y
\circ f'$. Because the Yoneda embedding is mono, such an $f'$ is
unique if it exists. Furthermore, each $f'_\Gamma(x)$ is characterized
by the following unique existence property:
\[ f'_\Gamma(x) \in_\Gamma f(x) \wedge \forall y \in Y. y \in_\Gamma f(x) \Rightarrow y \leq_\Gamma f'(x)\]
So we can equivalently say that $f$ is representable when for every $\Gamma, x : X_\Gamma$ that there exists a (necessarily unique) $r$ such that
\[ r \in_\Gamma f(x) \wedge \forall y \in Y. y \in_\Gamma f(x) \Rightarrow y \leq_\Gamma r \]
The only thing to prove is that such $r$s are necessarily monotone in $x$ (exercise)\footnote{this is the analogue of the representability to functoriality proof Pranav and Steven showed}.

Given a monotone function $f : Y \to X$ we can define a monotone
function\footnote{names here are not bad and should not be used in
Agda code} $\widetilde f : X \to \downsets Y$ by $y \in_\Gamma
\widetilde f(x) \iff f(y) \leq_\Gamma x$. We say $f$ has a
\textbf{right adjoint} when $\widetilde f(x)$ is representable.
Given a presheaf $A$, we say $X$ has \textbf{$A$-indexed meets} when the monotone function
\[ \Delta : X \to X^A \]
defined by
\[ \Delta_\Gamma(x)(\gamma,a) = x\gamma \]
has a right adjoint.
We can similarly define that $X$ has \textbf{a top element, or nullary meet} when the unique
\[ ! : X \to 1 \]
has a right adjoint. Note that this gives a morphism $\top : 1 \to X$ which externally is what we might call a \emph{section}
\[ \top : \forall \Gamma \to X_\Gamma \]
satisfying
\[ \top_\Gamma\gamma = \top_\Delta \]
and representability says that each $\top$ is a top element of
$X_\Gamma$. So requiring that $X$ has a top element is equivalent to
defining $X$ to be a functor form $C^{op}$ to the category of posets
with a top elements and morphisms that preserve them.

And define that $X$ has \textbf{binary meets} when
\[ \Delta : X \to X \times X \]
defined by
\[ \Delta_\Gamma(x) = (x,x) \]
has a right adjoint. Externally this is similar to the case for top,
we get a meet in each poset that is preserved by reindexing.

Now we can take all the above definitions about downsets and
representability and reformulate them dually for upsets. Or we can define
\[ \upsets X = (\downsets X^{op})^{op} \]
And with some careful op-ing re-use all the theory of representability
shown above. For instance we immediately get dual noitons of
$A$-indexed joins, nullary and binary joins.

Finally, let $A$ be a presheaf and $X$ be a poset with a top
element. Define a monotone function
\[ c : (X^{A \times A}) \to X^A \]
by
\[ c_\Gamma(f)(\gamma, a) = f_\Gamma(\gamma , (a, a)) \]
This should be easy to prove monotone. Note that the product presheaf
$A \times A$ is constructed pointwise.

% = : 1 → 𝓟↑ ((X^A) ^ A) that represents the upset P(x,y) is good when ∀ x. P(x,x)
Then we get $\widetilde c : X^A \to \upsets X^{A\times A}$. Further if $X$
has a top element then $X^A$ does as well which we call $\top : 1 \to
X^A$. We say $X$ has an $A$-equality predicate when
\[ \widetilde c \circ \top : 1 \to \upsets X^{A\times A} \]
is representable.

\section{First-Order Hyperdoctrines}

A first-order hyperdoctrine over a cartesian category $C$ consists of
\begin{enumerate}
\item (Propositional Logic) A biHeyting algebra $L$ internal to $\Set^{C^{op}}$.
\item (Universal Quantifiers) Such that for every $A \in C$, $L$ internally has $YA$-indexed meets
\item (Existential Quantifiers) Such that for every $A \in C$, $L$ internally has $YA$-indexed joins
\item (Equality) Such that $L$ has an internal equality function for $YA$
\end{enumerate}

\section{Syntax}

We can give a type theoretic syntax for first-order hyperdoctrines by
a combination of a simple type theory with terms $\Gamma \vdash M : A$
as well as intuitionistic propositional logic parameterized by a
context $\Gamma$ i.e. formulae $\Gamma \pipe \phi$ and proofs $\Gamma
\pipe \Phi \vdash \phi$. The universal quantifiers are presented as
follows:

\begin{mathpar}
  \inferrule{\Gamma,x:A \pipe \phi}{\Gamma \pipe \forall x:A. \phi}

  (\forall x:A. \phi)[\gamma] = \forall x:A. \phi[\gamma]
  
  \inferrule
  {\Gamma , x:A \pipe \Phi \vdash \phi}
  {\Gamma \pipe \Phi \vdash \forall x:A. \phi}

  \inferrule
  {\Gamma \vdash M : A}
  {\Gamma \pipe \forall x:A. \phi \vdash \phi[M/x]}
\end{mathpar}

Existential quantification is similar but you can eliminate it in any
context. Note that this relies on the model being a Heyting algebra,
we need some kind of CT structure thing to directly model these rules

\begin{mathpar}
  \inferrule{\Gamma,x:A \pipe \phi}{\Gamma \pipe \exists x:A. \phi}

  (\exists x:A. \phi)[\gamma] = \exists x:A. \phi[\gamma]
  
  \inferrule
  {\Gamma , x:A \pipe \Phi,\phi \vdash \psi}
  {\Gamma \pipe \Phi,\exists x:A.\phi \vdash \psi}

  \inferrule
  {\Gamma \vdash M : A}
  {\Gamma \pipe \phi[M/x] \vdash \exists x. \phi}
\end{mathpar}

Finally we have equality:

\begin{mathpar}
  \inferrule
  {\Gamma \vdash M : A \and \Gamma \vdash N : A}
  {\Gamma \pipe M =_A N}

  (M =_A N)[\gamma] = M[\gamma] =_A N[\gamma]

  \inferrule
  {}
  {\Gamma ,x:A \pipe \cdot \vdash x =_A x}

  \inferrule
  {\Gamma,x:A \pipe \Phi[y/x] \vdash \psi[y/x]}
  {\Gamma,x:A,y:A \pipe \Phi, x =_A y \vdash \psi}
\end{mathpar}

\section{Grothendieck Construction/Logical Relations}

Given a poset $X$ internal to presheaves on $C$ we can define a
category $\Sigma_C X$ as follows:
\begin{enumerate}
\item The objects are dependent pairs of $\Gamma \in C$ and $x \in X_\Gamma$
\item A morphism from $\Delta , y$ to $\Gamma , x$ is a $\gamma :
  C(\Delta,\Gamma)$ such that
  \[ y \leq_\Delta x\gamma \]
\item identity and composition are given by id/comp in $C$, verifying
  that the side-condition is preserved.
\end{enumerate}
We also clearly have a ``first-projection'' functor $\pi : \Sigma_C X \to C$.

Then we have the following theorems:
\begin{enumerate}
\item If $C$ has a terminal object and the poset $X$ internally has a
  terminal object then $\Sigma_C X$ has a terminal objects and $\pi$
  preserves it. Similarly for binary products/binary meets.
\item If $C$ has exponentials and the poset $X$ internally has Heyting
  implications and universal quantifiers over objects of $C$ then
  $\Sigma_C X$ has exponentials and $\pi$ preserves them.
\item If $C$ has an initial object and the poset $X$ internally has a
  bottom element then $\Sigma_C X$ has an initial object and $\pi$
  preserves is.
\item If $C$ has binary coproducts and the poset $X$ has internal
  binary joins and $X$ has existential quantifiers and equality for
  objects of $C$ then $\Sigma_C X$ has binary coproducts and $\pi$
  preserves them.
\end{enumerate}
Additionally in each of these cases the functor $\pi$ preserves the
structure (products/coproducts/exponentials).

\end{document}
